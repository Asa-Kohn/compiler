\documentclass{article}
\usepackage[utf8]{inputenc}

\usepackage{cite}

\title{
    Milestone 2: Teamwork Report\\
    \large Design Decisions + Contributions
}
\author{Alexander Iannantuono, Asa Kohn, and William Chien}
\date{March 15, 2020}

\usepackage{geometry}
\geometry{margin=1in}
\usepackage{hyperref}

\begin{document}

\maketitle

% FROM THE MILESTONE SPECIFICATIONS:
%
% BRIEFLY DISCUSS THE DESIGN DECISIONS YOU TOOK IN THE DESIGN AND IMPLEMENTATION
% OF YOUR SYMBOL TABLE / TYPE CHECKER. IF THERE ARE PARSING ISSUES THAT YOU
% IMPLEMENTED AS A WEEDING PHASE, DOCUMENT THEM HERE.
%
% - AN OVERVIEW OF THE SCOPING RULES USED
% - FOR EACH INVALID PROGRAM, THE CORRESPONDING TYPING RULE
% - SUMMARIZE HOW YOUR TEAM IS ORGANIZED AND WHAT EACH MEMBER CONTRIBUTED
% - RESOURCES CONSULTED

\section{Design and Implementation Decisions}

\subsection{Symbol Table}

% @asa, this is for later

\subsection{Typechecking}

Given that we used a lot of weaving in our symbol table, this made the
typechecking phase easier to design, although there was still a lot of
code to write. For each symbol in our symbol table, we include a pointer
that points to the node in the AST where the symbol is first introduced in the
program, i.e. where it is declared. By declared, we refer to the ``nearest''
declaration that can be found, as we support shadowing in GoLite.

In regards to writing the typechecker, we wrote short functions to
\textit{separate} the different kinds of declarations, statements or expressions.
This was so that the main function that would check if that given part of a program
type-checked and resulted in a cleaner and easier function.
Another example of this would be
for error printing. Most, if not all, of the errors have the same format of the
form:
    \[
        \verb|Error: (line #) Something went wrong with this (part).|
    \]

We say that it \textit{would be} given that we opted to copy-paste the code every time
that it was needed. This was due to the fact that some errors needed to output more
information than others. In terms of a function, we could implement that as
a linked list, but for something as basic as error reporting, it didn't seem
worth it to do that.

In terms of code, we wrote a few helper functions to be able to not have to
re-use code all the time, which is quite the standard practice, but it saved
us a lot of time. This includes functions that would add to the symbol table, or
add or remove a scope.
% add more after you put in stuff about the symbol table and stack implementation

For scoping, we used a linked list to be able to implement scoping in our compiler.
For example, ... %add more here

There were a few difficulties encountered while writing the typechecker. Firstly,
there are a lot of rules to keep track of. This is why we decided on separating
the checking of smaller parts of the program to type check, unless if it type
checked trivially, as then one could just refer to the smaller function.
Secondly, while implementing some of the typechecking rules, Alex realized that
we did not actually implement a boolean literal for the first milestone. We
just decided to use \verb$0$ and \verb$1$ in our C code in order to give them
a value. This became an issue as we needed to be able to do something with
identifiers, such as: \verb$a || b$. This would not work without some other
construct to describe the bool identifier, as \verb$2 || 3$ would be considered
valid, which is not allowed unless we set up some other notion of what we
define a \verb$bool$ to be considered in our compiler (as opposed to in GoLite).

% @will, talk about programs written in Q1 to test the typechecker

\section{Amending issues from the first milestone}

% included bool literal because we need it here
% fixing pretty printing invariance

% To discuss with @asa and @will

\section{Division of labour and team organization}

\subsection{Dividing the work}

 We'd like to note again that everyone in the group contributed equal amounts of
 respective work. We also all did testing together, this time it
 was done remotely via audio calling as we are all self-isolating at home.

\subsubsection{Asa}

% symbol table, fixing problems from first milestone

\subsubsection{Will}

% programs, helping others out

\subsubsection{Alex}

% report, typechecking, testing, fixing pretty printing invariance?
% repo organization

\subsection{Organization}

% Organization

\nocite{*}

% TODO!!
\bibliography{refs.bib}{}
\bibliographystyle{plain}

\end{document}
