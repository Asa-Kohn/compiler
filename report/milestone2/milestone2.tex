\documentclass{article}
\usepackage[utf8]{inputenc}

\usepackage{cite}
\usepackage{amsmath}

\title{
    Milestone 2: Teamwork Report\\
    \large Design Decisions + Contributions
}
\author{Alexander Iannantuono, Asa Kohn, and William Chien}
\date{March 3, 2020}

\usepackage{geometry}
\geometry{margin=1in}
\usepackage{hyperref}

\begin{document}

\maketitle

% FROM THE MILESTONE SPECIFICATIONS:
%
% BRIEFLY DISCUSS THE DESIGN DECISIONS YOU TOOK IN THE DESIGN AND IMPLEMENTATION
% OF YOUR SYMBOL TABLE / TYPE CHECKER. IF THERE ARE PARSING ISSUES THAT YOU
% IMPLEMENTED AS A WEEDING PHASE, DOCUMENT THEM HERE.
%
% - TODO AN OVERVIEW OF THE SCOPING RULES USED
% - TODO FOR EACH INVALID PROGRAM, THE CORRESPONDING TYPING RULE
% - TODO SUMMARIZE HOW YOUR TEAM IS ORGANIZED AND WHAT EACH MEMBER CONTRIBUTED
% - TODO RESOURCES CONSULTED

\section{Design and Implementation Decisions}

\subsection{Symbol table}

To build the symbol table, we do two passes through the AST. On the first pass,
we get an upper bound on the number of symbol records we'll need to store. On
the second pass, we store information about each symbol in one big array while
also building and freeing a parent-pointer stack. We check blank identifier uses
in this pass and link each use of a symbol to the record for it in the big
array.

One difficult part of this to get right was type declarations. The solution we
ended up with is label for each type in the AST indicating whether it's a type
literal, and if so which kind; a defined type; or an unresolved reference to a
type by name. In the pass where we build the symbol table, we also resolve all
of these unresolved references and replace them with the type the symbol refers
to.

Symbol table printing is done separately in another source file and in another
pass.

\subsection{Type checking}

Type checking was simpler than other parts of this milestone. There were no
major unexpected issues. Complicated scoping rules are handled by the symbol
table builder. Special handling of the blank identifier was added on afterwards,
so it isn't structured as cleanly as it might be. For a blank identifier, the
symbol table builder inserts null pointers instead of pointers to symbol
records, and the type checker checks for them where appropriate.
% @will, talk about programs written in Q1 to test the typechecker

\subsection{Weeding}

We also implemented the check for terminating statements in functions that
return values as a third weeding pass.

\section{Amending issues from the first milestone}

% TODO fixing pretty printing invariance
% TODO to discuss with @asa and @will

\subsection{String literal in lexer}
On line 204. Previously, our lexer would match string literals with newline characters as long as they are within the two double quotes. The issue is fixed where newline characters within two double quotes will not be matched.

\subsection{Int literal in lexer}
On line 199. Previously hexadecimal numbers are matched using:
\[
        \verb|0x[0-9a-fA-F]+|[0-9]+
\]
Now we use:
\[
        \verb|0x[0-9a-fA-F]+|[1-9][0-9]*|0[0-7]*
\]

\subsection{Rune \textbackslash{}' in lexer}
On line 188. \verb| '[\^\\']' | is added.

\subsection{Line number recording in parser}
\verb|$$->lineno = yylineno| on line 231 for TOK\_VAR is moved down two lines. Previously, we were recording line number prior to grouping, i.e. \verb| $$ = $3; |

\subsection{Three-part for loop in parser}
On lines 738-748, three-part for loop with empty condition is added.

\subsection{Weeding}
A typo on line 138 is fixed, where exp meant expstmt.

\subsection{Pretty printing}
traverse\_fields(VARS * f) added in pretty printing. On the other hand, segmentation faults occurred when null pointers are left unchecked, therefore, null checks are added to each pretty printing function.

\section{Division of labour and team organization}

\subsection{Dividing the work}

 %We'd like to note again that everyone in the group contributed equal amounts of
 %respective work. We also all did testing together, this time it was done
 %remotely via audio (and video) calling as we are all self-isolating at home.

 Asa decided to work on the symbol table and modeled it after what he did in his assignment. \\
 Alex decided to work on the typechecker and wrote primarily a boilerplate code for the typechecker file. \\
 William wanted to work on the test programs, as he previously wrote a majority of them in M1. \\
 Alex started to work on writing this report alongside the typechecker. However, due to an unexpected illness, William took over the report; and bravely enough, Asa took over the typechecker.

\subsubsection{Asa}

% TODO @asa: symbol table, fixing problems from first milestone
% TODO @asa: write about how you implemented your symbol table and typechecker in the second assignment
Asa has completed the symbol table by himself. He also picked up where Alex has left off in the typechecker. 

\subsubsection{William}

% TODO @will: programs, helping others out
% TODO @will: write about how you implemented your symbol table and typechecker in the second assignment
Over the fear of M1 for not being able to test our winnipeg compiler adequately robust and to catch a good number of bugs, Will has written roughly 200 invalid test files (non-distinct) and about 100 valid test files; and additionally, a \verb|testscript.sh| that allows us to run through a good number of test files against our compiler quickly. William tried to go through the goLite and blank identifier specifications slowly and hopefully in greater detail with additional research, then write out all the test files required to test our compiler. Though, whenever a test idea comes to mind, a file may be added to avoid forgetfulness; and since all test files are written over the span of weeks, some test cases may be non-unique or have been considered previously in a different file. In terms of the report, Will kept the majority of what Alex has left off, and filled in additional information when needed.

\subsubsection{Alex}

\textit{Alex wrote this section while he was still working on the project. He
  included comments indicating that he wasn't finished with it. Will and Asa
  have not modified it.}

% TODO? report
% TODO typechecking
% TODO testing
% TODO repo organization

In Alex's implementation of the symbol table, he used a bit of weaving, but
mainly decided to have every symbol be looked up and perform a linked list-like
search, which in hindsight is very wasteful in resources. For example,
given some symbol \verb$a$, his code would look through the current scope and
search in outer scopes until it found what it was looking for. This differs
in this milestone's implementation as here when we look at a symbol in say, some
expression, we have a pointer that points immediately to its declaration in
the code. This allows for much faster lookup, at the cost of an extra pointer
per identifier. Alex's typechecker was similar to that of this one, given
that he has written both his own and this group's with the eventual help of
Asa.

% TODO WRITE MORE ABOUT THIS, IT'S STILL WISHY WASHY
Alex spent most of his time working on the typechecking from boilerplate code
to the testing (hopefully fixing of bugs). As mentioned above, the report was
mainly written by him, except for the parts that involved the work of others.
Those parts were written by the members involved.

Another portion of this milestone that he decided to work on was fixing the pretty printing invariance from the first milestone.
He had been informed that a decent amount of the pretty printing that was tested during grading did not satisfy the following invariance:
\[
    \texttt{pretty(pretty(code)) = pretty(code)}
\]

After some time writing a script to do some basic invariance testing in a shell
file, Alex decided to find why some of the files did not respect the
invariance defined above. As it turns out, the majority of the issue was the
most basic bug that one finds in writing C code: not checking for a null pointer.
While traversing parts of the AST, Alex had forgotten to check for a null
pointer, causing the pretty printer to either cause a segmentation fault or to
(somehow) fail silently and stop printing. We're not sure if this is what
triggers the invariance test to yield a ``no''. If it's done by computer and
not by a human-TA, then this would make sense, as it would not print out everything that it needed to print. Another reason that caused the pretty
printing invariance was actually using the wrong function to print out fields
of a struct: it would print any set of fields as:
\[
    \texttt{f1 t1, f2 t2, ..., fk tk}
\]
Despite that not working for fields of arbitrary type. This yielded a syntax error as
soon as it went through to be pretty printed again, yielding the error. This was easily fixed by just putting them on new lines.

The last problem was due to how for-loops were printed. This yielded another
syntax error as it thought that an expression was somehow a function call in the
form of \verb$f(expr)$ where \verb$f := for$. This was due
to not correctly putting semicolons (which was only put if a certain part of a
for-loop was not null), and this was an easy fix as well.

\subsection{Organization}

% Organization

Organizing this time was a bit more challenging than for the first milestone. We originally preferred to meet in person, and have done so for a number of times.
However, given the ongoing/evolving global pandemic and school closure, we decided to work remotely at
our homes and communicate via various platforms: Matrix via Riot.im,
GitHub issues, email, text messaging, and online meetings.

Alex decided to use small to-do lists in the files that he was working on just
to keep track of the work he still had to do. Putting \verb|TODO: (thing)|
was also very helpful in case he missed something in the big files that he was
working on. For other parts of the project, such as writing this report, Alex
added small comments tagging either Will or Asa with \verb|@<name>: message| in
case there was something where their input was needed or Alex was requesting that
they possibly do that part.

% @will, @asa: please add what you used to stay organized for yourselves

\nocite{*}

% TODO fix the dates and references to remove old and put in new ones
\bibliography{refs.bib}{}
\bibliographystyle{plain}

\end{document}
