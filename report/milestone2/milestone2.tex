\documentclass{article}
\usepackage[utf8]{inputenc}

\usepackage{cite}
\usepackage{amsmath}

\title{
    Milestone 2: Teamwork Report\\
    \large Design Decisions + Contributions
}
\author{Alexander Iannantuono, Asa Kohn, and William Chien}
\date{Some date in the future, 2020}

\usepackage{geometry}
\geometry{margin=1in}
\usepackage{hyperref}

\begin{document}

\maketitle

% FROM THE MILESTONE SPECIFICATIONS:
%
% BRIEFLY DISCUSS THE DESIGN DECISIONS YOU TOOK IN THE DESIGN AND IMPLEMENTATION
% OF YOUR SYMBOL TABLE / TYPE CHECKER. IF THERE ARE PARSING ISSUES THAT YOU
% IMPLEMENTED AS A WEEDING PHASE, DOCUMENT THEM HERE.
%
% - TODO AN OVERVIEW OF THE SCOPING RULES USED
% - TODO FOR EACH INVALID PROGRAM, THE CORRESPONDING TYPING RULE
% - TODO SUMMARIZE HOW YOUR TEAM IS ORGANIZED AND WHAT EACH MEMBER CONTRIBUTED
% - TODO RESOURCES CONSULTED

\section{Design and Implementation Decisions}

\subsection{Symbol Table}

% @asa, this is for later

\subsection{Typechecking}

Given that we used a lot of weaving in our symbol table, this made the
typechecking phase easier to design, although there was still a lot of
code to write. For each symbol in our symbol table, we include a pointer
that points to the node in the AST where the symbol is first introduced in the
program, i.e. where it is declared. By declared, we refer to the ``nearest''
declaration that can be found, as we support shadowing in GoLite.

In regards to writing the typechecker, we wrote short functions to
\textit{separate} the different kinds of declarations, statements or expressions.
This was so that the main function that would check if that given part of a program
type-checked and resulted in a cleaner and easier function.
Another example of this would be
for error printing. Most, if not all, of the errors have the same format of the
form:
\[
        \verb|Error: (line #) Something went wrong with this (part).|
\]
We say that it \textit{would be} given that we opted to copy-paste the code every time
that it was needed. This was due to the fact that some errors needed to output more
information than others. In terms of a function, we could implement that as
a linked list, but for something as basic as error reporting, it didn't seem
worth it to do that.

In terms of code, we wrote a few helper functions to be able to not have to
re-use code all the time, which is quite the standard practice, but it saved
us a lot of time. This includes functions that would add to the symbol table, or
add or remove a scope.
% TODO add more after you put in stuff about the symbol table and stack implementation

For scoping, we used a linked list to be able to implement scoping in our compiler.
For example, ... % TODO add more here

% TODO talk about scoping rules used in the milestone

There were a few difficulties encountered while writing the typechecker. Firstly,
there are a lot of rules to keep track of. This is why we decided on separating
the checking of smaller parts of the program to type check, unless if it type
checked trivially, as then one could just refer to the smaller function.
Secondly, while implementing some of the typechecking rules, Alex realized that
we did not actually implement a boolean literal for the first milestone. We
just decided to use \verb$0$ and \verb$1$ in our C code in order to give them
a value. This became an issue as we needed to be able to do something with
identifiers, such as: \verb$a || b$. This would not work without some other
construct to describe the bool identifier, as \verb$2 || 3$ would be considered
valid, which is not allowed unless we set up some other notion of what we
define a \verb$bool$ to be considered in our compiler (as opposed to in GoLite).

% @will, talk about programs written in Q1 to test the typechecker

\section{Amending issues from the first milestone}

% TODO include bool literal because we need it here
% TODO fixing pretty printing invariance

% TODO to discuss with @asa and @will

\section{Division of labour and team organization}

\subsection{Dividing the work}

 %We'd like to note again that everyone in the group contributed equal amounts of
 %respective work. We also all did testing together, this time it was done
 %remotely via audio (and video) calling as we are all self-isolating at home.

 Asa had decided that he wanted to work on the symbol table and modeled it after
 what he did in his assignment\footnote{This will be explained in Asa's section.}.
 As a result of this, Alex decided that he would work on the
 typechecker and put in a lot of the boilerplate code for this. Alex was also
 going to write some of the programs, but that left Will with virtually nothing
 to work on by himself at first and therefore writing the programs was given to
 Will as a task. As Alex wrote the report last time and wanted to improve on his
 report writing skills, he decided to have another go at writing this
 one\footnote{He hopes that this report is more thorough and less ambiguous
 than the previous one.}.

\subsubsection{Asa}

% TODO @asa: symbol table, fixing problems from first milestone
% TODO @asa: write about how you implemented your symbol table and typechecker in the second assignment

\subsubsection{Will}

% TODO @will: programs, helping others out
% TODO @will: write about how you implemented your symbol table and typechecker in the second assignment


\subsubsection{Alex}

% TODO? report
% TODO typechecking
% TODO testing
% TODO repo organization

In Alex's implementation of the symbol table, he used a bit of weaving, but
mainly decided to have every symbol be looked up and perform a linked list-like
search, which in hindsight is very wasteful in resources. For example,
given some symbol \verb$a$, his code would look through the current scope and
search in outer scopes until it found what it was looking for. This differs
in this milestone's implementation as here when we look at a symbol in say, some
expression, we have a pointer that points immediately to its declaration in
the code. This allows for much faster lookup, at the cost of an extra pointer
per identifier. Alex's typechecker was similar to that of this one, given
that he has written both his own and this group's with the eventual help of
Asa.

% TODO WRITE MORE ABOUT THIS, IT'S STILL WISHY WASHY
Alex spent most of his time working on the typechecking from boilerplate code
to the testing (hopefully fixing of bugs). As mentioned above, the report was
mainly written by him, except for the parts that involved the work of others.
Those parts were written by the members involved.

Another portion of this milestone that he decided to work on was fixing the pretty printing invariance from the first milestone.
He had been informed that a decent amount of the pretty printing that was tested during grading did not satisfy the following invariance:
\[
    \texttt{pretty(pretty(code)) = pretty(code)}
\]

After some time writing a script to do some basic invariance testing in a shell
file, Alex decided to find why some of the files did not respect the
invariance defined above. As it turns out, the majority of the issue was the
most basic bug that one finds in writing C code: not checking for a null pointer.
While traversing parts of the AST, Alex had forgotten to check for a null
pointer, causing the pretty printer to either cause a segmentation fault or to
(somehow) fail silently and stop printing. We're not sure if this is what
triggers the invariance test to yield a ``no''. If it's done by computer and
not by a human-TA, then this would make sense, as it would not print out everything that it needed to print. Another reason that caused the pretty
printing invariance was actually using the wrong function to print out fields
of a struct: it would print any set of fields as:
\[
    \texttt{f1 t1, f2 t2, ..., fk tk}
\]
Despite that not working for fields of arbitrary type. This yielded a syntax error as
soon as it went through to be pretty printed again, yielding the error. This was easily fixed by just putting them on new lines.

The last problem was due to how for-loops were printed. This yielded another
syntax error as it thought that an expression was somehow a function call in the
form of \verb$f(expr)$ where \verb$f := for$. This was due
to not correctly putting semicolons (which was only put if a certain part of a
for-loop was not null), and this was an easy fix as well.

\subsection{Organization}

% Organization

Organizing this time was a bit more of a challenge than for the first milestone.
Given the ongoing and evolving global pandemic, we decided to work remotely at
our homes and communicate via various platforms: Matrix via Riot.im,
GitHub issues, email and text messaging.

Alex decided to use small to-do lists in the files that he was working on just
to keep track of the work he still had to do. Putting \verb|TODO: (thing)|
was also very helpful in case he missed something in the big files that he was
working on. For other parts of the project, such as writing this report, Alex
added small comments tagging either Will or Asa with \verb|@<name>: message| in
case there was something where their input was needed or Alex was requesting that
they possibly do that part.

% @will, @asa: please add what you used to stay organized for yourselves

\nocite{*}

% TODO fix the dates and references to remove old and put in new ones
\bibliography{refs.bib}{}
\bibliographystyle{plain}

\end{document}
