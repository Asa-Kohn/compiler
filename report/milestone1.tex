\documentclass{article}
\usepackage[utf8]{inputenc}

\usepackage{cite}

\title{
    Milestone 1: Teamwork Report\\
    \large Design + Implementation Decisions
}
\author{Alexander Iannantuono, Asa Kohn, and William Chien}
\date{February 21, 2020}

\usepackage{geometry}
\geometry{margin=1in}
\usepackage{hyperref}

\begin{document}

\maketitle

\section{Design and Implementation Decisions}

\subsection{Rationale of implementation tools chosen}

The first milestone as well as subsequent ones have and will be written in C
using flex and bison. Given that we all used this in prior coursework, it made
sense to go down this path of implementation tools. In the sections
that follow, we will detail the implementation decisions undertaken for each
component of our GoLite compiler required for this milestone.

\subsection{Scanning}
% Multiline comments (match)

Most of the scanning implementation was the same as that in the assignments
earlier in the semester. Given that GoLite has more keywords and built-in
features than MiniLang, there were of course more to account and scan for.

Multiline comments were a point of discussion and work on. We ended up deciding
to match the first \verb|\*| the first occurence of the closing token \verb|*/|
for the comment block, as GoLite does not support nested comments.
This was implemented using an elegant regular expression:

\[
    \verb|\/\*[^\*]*\*+([^\*\/][^\*]*\*+)*\/|
\]

As far as our testing goes, it appears to be functional.

% semicolon insertion (keep track of the last token)

For semicolon insertion rules, we decided to keep track of the last token prior
to seeing any of the tokens found in the insertion rules. This can be found in
the Go language documentation [[ref]].

% is there anything else?

\subsection{Parsing}
% function parsing when there are 0 and 1+ args
% func name () <--- %noassoc pragma

In order to parse functions, we decided to segregate those with no arguments and
those with non-zero arguments as different rules in the grammar.
This allowed for better legibility and hopefully
causes less issues when it comes to grammar rules which could result in
unwanted grammar productions. For the functions with zero arguments, we used the
\verb|%nonassoc| pragma on the \verb|(| and \verb|[| tokens. We gave this the
highest precedence as this needs to be considered first.
It appears as if that this could also be completed using
the \verb|%right| pragma, although it didn't
seem as elegant and appeared to be more work to implement correctly. The same
thing was done for expression lists of zero and non-zero size. This allowed for
easier reading and understanding of the grammar.

% chat with Asa

\subsection{Abstract Syntax Tree (AST)}

The AST was chosen to be quite simple and we used interleaving as our method
of implementation. It follows the same structure as the assignment although
there were a couple of differences as to what Alex and Will had done in their
respective solo assignments. In this part we decided to create many C structures
in order to keep the different kind of node types separated. The \verb|else if|
linking follows more of what Asa had done in his assignment.

% talk about this more

\subsection{Pretty Printing}
%TODO

\subsection{Implementation difficulties encountered}

Thankfully, there was not much difficulty in implementation as far as we can
tell, but time constraints induced some challenges.

\section{Division of labour and team organization}

\subsection{Dividing the work}

% speak to each individual so that I know exactly what they did

\subsubsection{Asa}

% lexer, parser
Asa was mainly in charge with writing most of the lexer and parser,
although decisions and discussions were done as a group.

\subsubsection{Will}

% programs, general, etc

Will had written most of the programs (both valid and invalid ones) to be
submitted for this milestone. He did some testing to make sure that the code was
correct and debated on what were `interesting' valid programs with Alex.
 He was responsible for organizing that part of the project to confirm it
 met the specifications of the first milestone.

\subsubsection{Alex}

% parser verification
% pretty printing (possibly) + testing?

Alex was mostly in charge with writing programs and verifying the work of
others. This included going over the bison source code file in an attempt to discover
\textit{elegant} ways of writing the required grammar for GoLite. He also took
charge in organizing the work and being in charge of keeping the repository in check.
By keeping the repository `in check' we mean that what has to be accomplished is
(or attempts to be) well established within the repository. This also includes
making sure
that everything is done as properly as possible. Finally, taking charge
of making pull requests and merging is a part of this task.

% Talk about testing once that actually gets done.

\subsection{Organization}

% mention what tools we use to be productive, etc
% how often we met etc

In order to be productive, we decided to set up certain communication channels.
This was done using \href{https://matrix.org/}{Matrix} which is an open source
alternative to Slack. Further, some markdown files were created to do basic
TODO lists, but given the time frame, it wasn't as successful -- each team
member had their own list of things to focus on instead.

%\bibliography{refs}{}
%\bibliographystyle{plain}

\end{document}
