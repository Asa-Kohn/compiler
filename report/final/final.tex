\documentclass{article}
\usepackage[utf8]{inputenc}

\usepackage{cite}
\usepackage{amsmath}

\title{COMP520 GoLite Project: Final Report}
\author{Alexander Iannantuono, Asa Kohn, and William Chien}
\date{May 1, 2020}

% NOTE target page size is ~20 pages
% TODO find out what


\usepackage{geometry}
\geometry{margin=1in}
\usepackage{hyperref}

% remove this when done
\usepackage{xcolor}

\begin{document}

\maketitle

\section{Introduction}

\subsection{Overview of Go}

The Go programming language is a statically typed, compiled prgramming language that gives programmers the speed of C with the legibility of high level languages like Python. It also allows for higher level constructs ``out of the box'' such a memory safety and concurrency, which is not something that one would find in a language such as C. It is increasingly popular in many software projects and is developed by three individuals \textcolor{red}{(names?)}, the legendary Ken Thompson among them. The language is also open source software, which aligns with our beliefs.

Due to lack of ``out of the box'' features in GoLite, programs are generally longer to write. This would be the case when a program requires
using a feature that is not supported in GoLite, but is in Go. At that point, one would have to implement the feature that is required themselves in their code and their source is therefore lengthier (assuming they are not using a library).

% NOTE As a side note, a language would less features for high level coding would generally have more libraries to make those features available from a third party, instead of the developers or maintainers of the language. (right?)

\subsection{Go versus GoLite}

The possibility of implementing a fully functional compiler for a language such as Go is infeasible given the time constraints of a semester\footnote{Minus two weeks for this year's course.}. Therefore, a subset of the Go language was introduced to us, and was given the name of GoLite.

\subsubsection{Notable changes}

% semicolon rules

% program structure
    % no packages  --> no imports
    % no constant declarations at the top level
    % no type aliasing
    % function declarations (a lot)
    % only 5 basic types: int float64 bool rune string
    % no slice literals
    % no array literals
    % no expression in array size
    % no anon members
    % no support tags
    % no struct literals
    % only print() and println() for output
    % return accepts zero or one expression only, not more
    % no type switch statements
    % only three type of for-loops: inf, while and three part
    % break + continue unlabelled forms
    % only 4 unary operators
    % no trailing comma in function calls
    % no slice expressions
    % simpler append()
    % primitive type casting

\subsection{Structure of Report}

For this report's structure, we follow the general structure that is presented in the document given by the course's instructor. That being said, we write about the overview, design decisions and all the testing done \textit{for each component} of the compiler as three seperate parts. We did this instead of writing about all three parts (overview, design decisions, testing) for each component separately since the design decisions and testing done in components up in the pipeline influenced modifications to decision decisions made and to be aware of in the future. This applies to testing as well: cases that we missed and to make sure to look into those cases in the future. What we've decided to do allows for explanation of how all components interact with each other in specific contexts (rather that going back and forth).


% TODO Add more

For the conclusions, each member of the group has written their own part personally, and we indicate their name so it is clear who's experience is being read.

% TODO Add more

The contributions section was agreed upon by all three of us before submission so that everyone in the group is on the same page as each other in terms of project involvement.


\section{Language and Tool Choices}

\subsection{Language and Tools}

As mentioned in the first milestone, we opted for sticking with using C and the flex, bison tool-chain to implement the compiler for the project. Despite having another tool-chain available: SableCC, all three of us had already used the former tool-chain to implement the two personal assignments and agreed that we did not want to spend extra time learning a new set of tools. Given this choice, we spent a lot of time debugging as one usually does with C. Many segmentation fault and bug fixes were done to get parts working. While it was challenging to find the bugs, we managed to sort a lot of them out to complete the milestones with Asa's remarkable ability to trace bugs in C using valgrind.

As for other tools that we used, we communicated mainly through Riot.im,[link?] a client for the Matrix communication protocol. Otherwise, it was through email, text messages or the occasional phone call if the first platform was not active. Alex and Asa (?) and Will (?) had personally hoped that we could use GitHub issues to track problems to fix, and use to-do markdown files to keep track of tasks. Unfortunately, these efforts were futile and were abandoned during the first milestone.

\subsection{Target Language}

% TODO: This section.

\textcolor{red}{TODO}

\section{Compiler Components}

% NOTE no pretty printing?

\subsection{Scanner}

% TODO I didn't do this (except for a regex or two), who did @asa @will

% Implemented using flex
% Rules for tokens (basic)
% REGEX for everything else (put them in)
    % identifiers
    % literals
    % comments
    % basic tokens and keywords
% some of the REGEXes could have been doing using CFGs that were easier to read
    % some bugs were caused because of this, that we had to fix in m2
% rejecting things that aren't supported in GoLite (put them in)

\subsection{Parser}

% TODO I didn't do this except for testing, who did? @asa @will

\subsection{Abstract Syntax Tree (AST)}

% TODO @asa @will

\subsection{Weeder}

% TODO @asa @will

\subsection{Symbol Table}

% TODO @asa

\subsection{Typechecker}

% TODO ~@alex @asa

\subsection{Code Generation}

% TODO @asa @will

\section{Conclusion}

\subsection{Experiences}

\subsubsection{Alex}

% TODO @alex

Given the unprecedented global situation that we are all facing, along with
medical complications that worsened over the course of the semester, working on the project became increasingly challenging to do. That being said, I had a great time doing what I could: learning how something like the Go programming language works from source to machine
code is something that interests me greatly. This might stem from the tinkerer-side of me, which was originally had started by taking mechanical devices apart to see how they functioned. A project like this allowed for me to discover how a compiler works, and I am glad that I was able to do this.

While many at McGill or even university dislike group projects because of lack
of coordination and communication from fellow group members, it's peculiar for
me being the one that has done the least amount of work (this is my own
observation), given that in past projects during my academic career I felt like I was doing a reasonably amount to the majority of it.
That being said, I cannot begin to express the appreciation that I have for the
understanding and compassion that Asa and Will had showed throughout this project, as well as Alex's (instructor) understanding of my situation. I personally think that we got along really well and had a great time despite the stressful crunch times that we had been in. In terms of group projects, I would keep the group component of the project as it is, because while many find it difficult, we live in a society where collaboration and cooperation is vital. As a somewhat strange note: oddly enough, governments even seem to struggle with these things as this pandemic continues to unfold.

As for the actual content of the project, I would just say that perhaps more instructions could be given on certain components of the compiler. I understand that this is a 500 level course, and that we are (mostly) adults that should be able to forecast our abilities to do the work, although the project is inherently demanding as is the course. Perhaps there could be an easing of certain constructs within GoLite that could be removed or modified, which would make the implementation of the compiler slightly more straightforward.

\subsubsection{Asa}

% TODO @asa

\subsubsection{Will}

% TODO @will

\section{Contributions}

\subsection{Alex}

To be completely honest, I do not feel comfortable saying that I did anything near a third of the work for this project. I definitely helped out in many aspects throughout the implementation and project, but I did not lead on any development of the components of the compiler besides the pretty printer, which is not really a component of the compiler. That being said, I helped Will write tester code and did testing of the components when it was necessary to do so. I took care of the pretty printer and fixing any bugs that I missed during the first milestone submission. I mainly worked on the reports (including this one), with the exception of the report for the second milestone. This is due to me falling ill and there were modifications to our implementation that were not reflected in what I had originally written. I wrote a couple of the initial regular expressions (i think?) for the scanner. I wrote the boilerplate code for the typechecker, but after looking at the submission, it doesn't appear as if there was much that was kept in the submission for the second milestone.

The difficulty during the fourth milestone is the limited monitor usage that I have to be careful about, otherwise I develop a migraine and then I am unable to do work for most of the day or until the pain subsides. Therefore, I was fortunate enough to be able to work on parts of the project that are relatively independent of other components (such as this report) or the benchmarks that are required for the final milestone. I admit that it is very limited work, but so are the options to contribute that are available at this time.

% I am hoping to write the benchmarks for the fourth milestone, but this did not occur

\subsection{Asa}

% TODO @asa

\subsection{Will}

% TODO @will

\nocite{*}

\bibliography{refs.bib}{}
\bibliographystyle{plain}

\end{document}
